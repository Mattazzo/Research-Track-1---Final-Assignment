Matteo Azzini 4475165

The assignment requires to develop a software architecture for the control of the robot in the environment. The software will rely on the move\+\_\+base and gmapping packages for localizing the robot and plan the motion.

The architecture should be able to get the user request, and let the robot execute one of the following behaviors (depending on the user\textquotesingle{}s input)\+: \begin{DoxyVerb}1 - Move randomly in the environment, by choosing 1 out of 6 possible target positions: [(-4,-3);(-4,2);(-4,7);(5,-7);(5,-3);(5,1)]

2 - Directly ask the user for the next target position (checking that the position is one of the possible six) and reach it

3 - Start following the external walls

4 - Stop in the last position

5 - Change the planning algorithm from dijkstra (move_base) to the bug0 and viceversa
\end{DoxyVerb}


If the robot is in state 1, or 2, the system should wait until the robot reaches the position in order to switch to the state 3 and 4 or to change the planning algorithm.

\subsubsection*{Content of the package}


\begin{DoxyItemize}
\item {\bfseries docs} \+: It\textquotesingle{}s a folder containing online and offline documentation about this package
\item {\bfseries scripts} \+: It\textquotesingle{}s a folder containing all Python executable used in this package
\item {\bfseries worlds} \+: It\textquotesingle{}s a folder that define the world within the robot should move
\item {\bfseries launch} \+: It\textquotesingle{}s a folder containing all file with .launch extension needed to launch the simulation
\item {\bfseries urdf} \+: It\textquotesingle{}s a folder containing all information and description about the robot
\item {\bfseries param} \+: It\textquotesingle{}s a folder containing all file with .yaml extension neede to set parameters for the simulation
\item {\bfseries config} \+: It\textquotesingle{}s a folder containing a file useful for simulation tools
\item {\bfseries rosgraph.\+png} \+: It\textquotesingle{}s an image which graphically represent how nodes comuinicate and which topics they use
\item {\bfseries C\+Make\+List.\+txt} \+: It\textquotesingle{}s a file with informations about the compilation
\item {\bfseries package.\+xml} \+: It\textquotesingle{}s a file with informations about the compilation
\end{DoxyItemize}

In particular the folder \textquotesingle{}scripts\textquotesingle{} contain\+:
\begin{DoxyItemize}
\item \hyperlink{robot__controller_8py}{robot\+\_\+controller.\+py} \+: Principal node that control robot behaviour
\item \hyperlink{user__command__server_8py}{user\+\_\+command\+\_\+server.\+py} \+: Implement the server to take user command
\item \hyperlink{random__target__server_8py}{random\+\_\+target\+\_\+server.\+py} \+: Implement the server to set a random target for the robot
\item \hyperlink{user__target__server_8py}{user\+\_\+target\+\_\+server.\+py} \+: Implement the server to set a target for the robot taken from user input
\item \hyperlink{bug__m_8py}{bug\+\_\+m.\+py} \+: Implement a server to execute bug0 algorithm(professor version slightly modified)
\item \hyperlink{wall__follow__service__m_8py}{wall\+\_\+follow\+\_\+service\+\_\+m.\+py} \+: Implement a server to let robot follow the wall, used also in bug0 algorithm (professor version)
\item \hyperlink{go__to__point__service__m_8py}{go\+\_\+to\+\_\+point\+\_\+service\+\_\+m.\+py} \+: Implement a server to let robot go to a desired point, used also in bug0 algorithm (professor version)
\end{DoxyItemize}

\subsubsection*{How to run the simulation}

In order to run the simulation there is a launch file that can be launched with the followig command\+:


\begin{DoxyCode}
1 roslaunch final\_assignment final\_assignment.launch 
\end{DoxyCode}


It includes launch file for Gazebo and Rviz graphic user interfaces, launch file for move base algorithm and all necessary parameters and nodes to execute the simulation.

\subsubsection*{Robot behaviours}

When the simulation is launched the robot is spawned in the pre-\/built environment waiting for a command from user, once the command is given the robot can perform 5 different behaviour\+: \begin{DoxyVerb}1 - A random target is given from the random_target_server and the robot try to reach it following the active algorithm (by default is move base).
    In the maintime the user can insert the next command to be execute, once the robot has achieved the target, informs user printing on terminal and execute the next command.

2 - The robot requires a target by user thanks to user_target_server, once user has inserted it the robot try to reach it with the active algorithm.
    When the robot reach the target, it informs user and hask him for a new command.

3 - Robot starts following the wall, in the maintime the user can insert the next command to be executed.

4 - Robot is stopped and asks user for the next command.

5 - Moving algorithm is switched and robot asks for the next command. 
\end{DoxyVerb}


Behavior 3,4,5 can be executed only when robot has reached the target in 1 or 2, not during the execution.

Attention\+: move\+\_\+base has a recovery behavior for unreachable target, bug0 doesn\textquotesingle{}t, so I have implemented a timeout to avoid robot tryng to reach unfiesable targets, when the timer expires the robot stop itself and ask for a new command from user.

\subsubsection*{About software architecture}

I decided to have 4 parameters (target coordinates x and y, command inserted by user, a flag to check if the target is reached) in order to allow all nodes to check and modify them. When move base algorithm is active all behaviors are managed by the node \hyperlink{robot__controller_8py}{robot\+\_\+controller.\+py}, which calls user\+\_\+commmand\+\_\+server.\+py when it\textquotesingle{}s necessary to get a command. Once the command is inserted by the user, the node activates the corrispondent service to execute the desidered behavior, so it interacts with all services. Instead when bug0 is active the node \hyperlink{robot__controller_8py}{robot\+\_\+controller.\+py} is always responsable to call service to get user command and execute command 3,4,5 but for behavior 1 and 2 is the node \hyperlink{bug__m_8py}{bug\+\_\+m.\+py} that call services to get a target (randomly or from user). \hyperlink{bug__m_8py}{bug\+\_\+m.\+py} is also responsable to call wall\+\_\+follow\+\_\+service.\+py and go\+\_\+to\+\_\+point\+\_\+service.\+py to execute the bug0 algorithm to achieve the target and it is its duty also to set the timer for the recovery behavior (when robot try to reach unfeasible target) and to stop the robot wen the timer expires.

For a graphic representation of the software achitecture, please check final\+\_\+assignment\+\_\+rosgraph.\+png.

\subsubsection*{System limitations and possible improvements}

{\bfseries Limitations}\+:
\begin{DoxyItemize}
\item Sometimes the map can change position during robot motion, this obviously creates problem for robot localization
\item Sometimes the input is not detected, so when you press a number and it he acquires a null character
\item Sometimes it seems that Docker change scheduling order of my scripts unreasonably (so in very rarely cases input for commands overlaps input for target)
\end{DoxyItemize}

I think that these are problems related to Docker execution on my PC, but I cannot be sure of this.

{\bfseries Possible improvements}\+:
\begin{DoxyItemize}
\item There is a print about laser scanner, probably due to move base nodes, I would delete it to have a cleaner output on the terminal.
\item Find a more efficent recovery behavior for bug0 instead of the timer.
\item Print robot position or distance from the target when robot is moving. 
\end{DoxyItemize}